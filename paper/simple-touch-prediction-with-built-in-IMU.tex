\documentclass[sigchi-a, authorversion,table]{acmart}
\usepackage{booktabs} % For formal tables
\usepackage{ccicons}  % For Creative Commons citation icons
\usepackage{tabularx}
\usepackage{multirow}
\usepackage{multicol}
\usepackage{xcolor}
\usepackage{dcolumn}
\usepackage{url}
\newcolumntype{d}[1]{D{.}{.}{#1}}
\usepackage[capitalise,nameinlink,noabbrev]{cleveref}                 
\usepackage[colorinlistoftodos]{todonotes}
%\setlength{\marginparwidth}{2,5cm}

%Cleverref works using this
\setcounter{secnumdepth}{2}

% Copyright
\setcopyright{none}
%\setcopyright{acmcopyright}
%\setcopyright{acmlicensed}
%\setcopyright{rightsretained}
%\setcopyright{usgov}
%\setcopyright{usgovmixed}
%\setcopyright{cagov}
%\setcopyright{cagovmixed}


% DOI
\acmDOI{}

% ISBN
%\acmISBN{123-4567-24-567/08/06}

%Conference
\acmConference[FIS'18]{Fachpraktikum Interaktive Systeme}{July 2018}{Stuttgart, Germany}
\acmYear{2018}
\copyrightyear{2018}

\acmPrice{1000.00}

%\acmBadgeL[http://ctuning.org/ae/ppopp2016.html]{ae-logo}
%\acmBadgeR[http://ctuning.org/ae/ppopp2016.html]{ae-logo}

\begin{document}
\title{Simple Touch Prediction with Built-In IMUs}

\author{Benedict Steuerlein}
\affiliation{%
  \institution{University of Stuttgart}
  \city{Stuttgart}
  \country{Germany} }
\email{st111340@stud.uni-stuttgart.de}

\author{Felix Bühler}
\affiliation{%
  \institution{University of Stuttgart}
  \city{Stuttgart}
  \country{Germany} }
\email{st117123@stud.uni-stuttgart.de}


% The default list of authors is too long for headers.
\renewcommand{\shortauthors}{F. Author et al.}


%
% The code below should be generated by the tool at
% http://dl.acm.org/ccs.cfm
% Please copy and paste the code instead of the example below.
%


\begin{CCSXML}
<ccs2012>
 <concept>
<concept_id>10003120.10003121.10003122.10003334</concept_id>
<concept_desc>Human-centered computing~User studies</concept_desc>
<concept_significance>500</concept_significance>
</concept>
<concept>
<concept_id>10003120.10003138.10003141</concept_id>
<concept_desc>Human-centered computing~Ubiquitous and mobile devices</concept_desc>
<concept_significance>500</concept_significance>
</concept>
</ccs2012>
\end{CCSXML}

\ccsdesc[500]{Human-centered computing~User studies}
\ccsdesc[500]{Human-centered computing~Ubiquitous and mobile devices}
\begin{abstract}
Lorem ipsum dolor sit amet, consectetuer adipiscing elit. Aenean commodo ligula eget dolor. Aenean massa. Cum sociis natoque penatibus et magnis dis parturient montes, nascetur ridiculus mus. Donec quam felis, ultricies nec, pellentesque eu, pretium quis, sem. Nulla consequat massa quis enim. Donec pede justo, fringilla vel, aliquet nec, vulputate eget, arcu. In enim justo, rhoncus ut, imperdiet a, venenatis vitae, justo. Nullam dictum felis eu pede mollis pretium. Integer tincidunt. Cras dapibus. Vivamus elementum semper nisi. Aenean vulputate eleifend tellus. Aenean leo ligula, porttitor eu, consequat vitae, eleifend ac, enim. Aliquam lorem ante, dapibus in, viverra quis, feugiat a, tellus. Phasellus viverra nulla ut metus varius laoreet. Quisque rutrum. Aenean imperdiet. Etiam ultricies nisi vel augue. Curabitur ullamcorper ultricies nisi. Nam eget dui. Etiam rhoncus. Maecenas tempus, tellus eget condimentum rhoncus, sem quam semper libero, sit amet adipiscing sem neque sed ipsum. Nam quam nunc, blandit vel, luctus pulvinar
%See: \url{https://users.ece.cmu.edu/~koopman/essays/abstract.html}
\todo{noch ein bis zwei doofe keywords finden}
\end{abstract}


\keywords{Touch prediction; smartphone; sensors; regression; machine learning; .}

\maketitle



\graphicspath{{./pictures/}}
\begin{marginfigure}
	\includegraphics[height=\linewidth]{cross}
	\caption{Touch task with one cross displayed.\newline}
	\label{fig:touchtask}
\end{marginfigure}

\begin{marginfigure}
	\includegraphics[height=\linewidth]{fitts}
	\caption{Fitts Law task with a progress bar displaying the current progress.}
	\label{fig:fittstask}
\end{marginfigure}

\section{Introduction}
\label{sec:intro}
Touch is the preferred input method on smartphones today. 
Current research and manufacturers are constantly trying to improve and enhance the interaction on smartphones. 
Enhancing smartphones with new, rich interaction methods allows users to operate their phone faster and more accurate, thereby increasing the usability and user-experience.
One way can be the extension of interactable space with interaction possibilities on the back of device (BoD) \cite{Le2016,Baudisch2009,Le2017}.
E.g. unlocking the phone by using gestures on the backside, or explicit touches on the backside to reach for unreachable targets on the front side may be possible applications of this technique.
However, some of these solutions do not conform with the form factors and weights of ordinary phones \cite{Corsten2017,DeLuca2013,Wolf2012}.
Another way to extend interaction is the introduction of additional touch gestures and touch recognition on touchscreens \cite{Le:2018:PalmTouch, Holz2015,Guo2015}.
Previous work tried predicting touch positions on the touchscreen based on sensor data from either built-in or additional sensors \cite{Le2017:Predic,MohdNoor2016}.\\
Accurately predicting touch positions could offer many use-cases:

\textbf{Preloading Content}: Preloading certain content that a user might request in the near future, e.g. a web page, reduces the waiting time and thus improves the user experience. 
	Removing the latency and thereby getting rid of the delay between certain actions greatly enhances the usability of a system. 
	However, when navigating on websites and preloading content from small links a high and precise accuracy is required. 
	Inaccurate prediction would require to preload more data around the predicted point.

\textbf{Highlighting Objects}: When navigating through folders, predicted future positions of a touch can be used to highlight information about the to be touched folders. 
	Navigation through a picture gallery on the phone can slightly enlarge the to be touched pictures to give a small preview, maybe showing picture information of where and when the picture was taken.
	The accuracy required for this approach does not necessarily have to be very high as elements are usually larger than small weblinks.

Despite its novelties, accuracy of some of these solutions has been low. In this paper, motivated by related work and its unexploredness, we present a machine learning model for predicting touches on smartphones using only the phones' built-in sensors.



\section{Related Work}

Motivate your project by reporting about related work and common goals.
\section{Data Collection Study}
We conducted a data collection study to gather IMU data while performing touches on a smartphone.
Our collected data set consists of 6 smartphone sensors that were sampled while participants performed successive touches on the smartphones front side. 
For our data collection study we used a repeated-measures design with one independent variable: \textsc{phone}, which was counterbalanced using Latin Balanced squares. 
The total amount of conditions was: \textsc{phone} $ = 4$.

\subsection{Apparatus}
Our dataset was generated using four different sized smartphones on which participants had to perform a certain amount of touches (for further information see \cref{sec:tasks}).
The phones we used were a Samsung S3 Mini, a Samsung S4, a Google Nexus 5X, and a Motorola Nexus 6.
For more details about the devices used in the study see \cref{tab:devices}.
Our used phone sizes range from $ 4^{\prime\prime} $ (S3) to $ 6^{\prime\prime} $ (N6). 
Using phones of these sizes we were able to cover the sizes of everyday smartphones, including some high-end devices and create a generalizable machine-learning model.



\subsection{Tasks}
\label{sec:tasks}
\begin{marginfigure}
	\vspace{-3.5cm}
	\includegraphics[width=1\marginparwidth]{imu-cnn-korrekt}
	\caption[Model architecture]{Model architecture of the neural network of general models for predicting touches based on phones' IMU.}
	\label{fig:model_architecture}
\end{marginfigure}
For our data collection study participants had to touch points displayed as crosses in a 16 $ \times $ 9 grid on the touchscreen (see \cref{fig:touchtask}). 
We decided to create the crosses the same as the aspect ratios of the mobile phones.
To achieve a high variance, we randomized the positions of the crosses within all the cells.
To avoid sequential effects, we randomized the order in which the crosses were displayed.
There were a total of 3 repetitions, resulting in a total  of $ 16 \times 9 \times 3 = 432 $ touches on one device.

Between two touches our study participants had to perform a simple \textit{Fitts' Law task} (see \cref{fig:fittstask}). 
Here participants had to drag a filled rectangle into a dashed contour of a rectangle.
This task was mainly implemented to reset the participants grip to the bottom half of the device.
Because a previous shifted grip of the hand to the upper half of the phone influences the recorded sensor data when reaching for the next target in the lower half and vice versa.
\subsection{Procedure}
Participants were either invited within the course \textit{FIS'18} or orally.
All appointments were discussed orally.
After participants have arrived they signed a consent form, and we continued measuring their hand length.
We asked the participants to take a seat on a chair without armrests and explained the study procedure and its sense.
We handed out the first phone accordingly to the balanced Latin Square order. 
After participants finished the tasks (see \cref{sec:tasks}) on the first phone, we asked them if they need a short recovery break and then continued with the next phones.
The study duration was 54 minutes on average.

\subsection{Participants}
We invited 20 right-handed fellow students as participants (5 female).
Their age ranged between 21 and 27 ($ M=24.25$ , $SD=1.58 $). 
We measured the hand length of participants. 
The size was measured from the tip of the middle finger to the wrist crease with fingers stretched out.
Hand lengths ranged from $16.0cm$ to $21.3cm$ ($M=19.3cm$ , $SD=1.47cm$).
Our measured data covers samples from the 5th and 95th percentile of the anthropometric data reported in previous work \cite{Poston}.  

\section{Results}
\begin{margintable}
%	\vspace{-5cm}
	\centering
	\begin{tabularx}{1\marginparwidth}{Xd{2.1}d{2.1}d{1.1}d{1.1}}
		\toprule
		&
		\multicolumn{1}{c}{\multirow{2}{*}{\shortstack[c]{\textbf{S3M} \\ \textbf{0ms}}}}&    
		\multicolumn{1}{c}{\multirow{2}{*}{\shortstack[c]{\textbf{S4}\\ \textbf{0ms}}}} &
		\multicolumn{1}{c}{\multirow{2}{*}{\shortstack[c]{\textbf{N5X}\\ \textbf{0ms}}}} &
		\multicolumn{1}{c}{\multirow{2}{*}{\shortstack[c]{\textbf{N6}\\ \textbf{0ms}}}} \\
		\\
		\midrule
		%                	 Touch  33ms  66ms  
		RF & 28.11 & 34.76 &36.38 & 41.4 \\ \cmidrule{1-5}
		DT & 28.81 & 35.58 &37.03 & 42.56 \\ \cmidrule{1-5}
		KNN& 28.67 & 34.61 &36.36 & 42.91 \\ \cmidrule{1-5}
		GP & 53.35 & 66.83 &69.53 & 79.06 \\ \midrule
		
		%		\multicolumn{1}{l}{\multirow{2}{*}{\shortstack[c]{\textbf{Regressor}}}}&
		&\multicolumn{1}{c}{\multirow{2}{*}{\shortstack[c]{\textbf{S3M} \\ \textbf{33ms}}}}&    
		\multicolumn{1}{c}{\multirow{2}{*}{\shortstack[c]{\textbf{S4}\\ \textbf{33ms}}}} &
		\multicolumn{1}{c}{\multirow{2}{*}{\shortstack[c]{\textbf{N5X}\\ \textbf{33ms}}}} &
		\multicolumn{1}{c}{\multirow{2}{*}{\shortstack[c]{\textbf{N6}\\ \textbf{33ms}}}} \\
		\\
		\midrule
		%                	 Touch  33ms  66ms  
		%                	 Touch  33ms  66ms  
		RF & 27.91 & 34.29  & 36.05  & 40.45 \\ \cmidrule{1-5}
		DT & 28.39 & 35.1   & 35.03  & 41.52 \\ \cmidrule{1-5}
		KNN& 28.41 & 34.31  & 35.92  & 42.13 \\ \cmidrule{1-5}
		GP & 52.89 & 66.82  & 69.52  & 79.05 \\ 
		\midrule
		%		\multicolumn{1}{l}{\multirow{2}{*}{\shortstack[c]{\textbf{Regressor}}}}&
		&\multicolumn{1}{c}{\multirow{2}{*}{\shortstack[c]{\textbf{S3M}  \\ \textbf{66ms}}}}&    
		\multicolumn{1}{c}{\multirow{2}{*}{\shortstack[c]{\textbf{S4} \\ \textbf{66ms}}}} &
		\multicolumn{1}{c}{\multirow{2}{*}{\shortstack[c]{\textbf{N5X} \\ \textbf{66ms}}}} &
		\multicolumn{1}{c}{\multirow{2}{*}{\shortstack[c]{\textbf{N6} \\ \textbf{66ms}}}}\\
		\\
		\midrule
		%                	 Touch  33ms  66ms  
		RF &27.54 & 33.58  & 35.24  & 40.18 \\ \cmidrule{1-5}
		DT &27.9  & 32.74  & 34.67  & 39.54 \\ \cmidrule{1-5}
		KNN&28.1  & 33.99  & 35.93  & 41.62 \\ \cmidrule{1-5}
		GP &44.65 & 64.83  & 67.88  & 76.18 \\ 
		\bottomrule
	\end{tabularx}%
	\caption[Baseline data]{\small Average euclidean distances (mm) for baseline regressors from scikit-learn\protect\footnotemark.}
	\label{tab:baseline}
\end{margintable}
\footnotetext{\url{https://scikit-learn.org/stable/supervised\_learning.html\#supervised-learning}}





\subsection{Data Set \& Preprocessing}
\label{sec:prepro}
Look at \cref{tab:baseline}
We recorded a total of 1079 minutes of sensor data from the \textit{touch} and \textit{Fitt's Law task} task. 
%Since we are only interested in sensor changes during the process of reaching for a target and touching it and the purpose of the \textit{Fitt's Law task} was only for reseting the participants grip to the lower half of the phone our first step in our preprocessing pipeline was \texttt{cutting} out the sensor fragments of the \textit{Fitt's Law task}.
Due to the high sampling rate of the sensors we first removed occurring duplicates by keeping the last sensor value and timestamp.
We then up-sampled the sensors to 333.33 Hz, resulting in 1 sample given every 3ms. 
Finally, we saved 100 samples before each touch resulting in a total of 3.456.000 samples.

Up-sampling the sensors to 1 sample every 3ms resulted in a discontinuous function for each sensor axis. 
We have therefore tried applying several smoothing procedures.

\subsection{Neural Network Structure}


\footnote{\url{https://05.jupyter.interactionlab.io/user/beneste/tree/fapra-imu}}


%Report about your model. No source code!
%
%Report about the validation dataset / validation study. 
\section{Discussion}
\begin{marginfigure}
	\vspace{-8cm}	
	\centering
	\includegraphics[height=\linewidth]{ellipses_single}
	\caption{Elliptic envelope plot for single models.\newline}
	\label{fig:ell_single}
\end{marginfigure}
\begin{marginfigure}
	\includegraphics[height=\linewidth]{ellipses_general}
	\caption{Elliptic envelope plot for general models.}
	\label{fig:ell_general}
\end{marginfigure}
Motivated by related work, we conducted a study where participants performed touches on smartphones during which we sampled the internal sensors. 
A combination of a neural network and accelerometer and gyroscope values from the internal sensors allowed us to predict future touches on the smartphone. 
We have focused on showing the possibility of the applicability of our approach as using the IMU is feasible for ordinary usage.

Our results show that touch prediction with neural networks based on internal sensor values is possible, with fairly high precision accuracy.
For increasing prediction times prediction errors increase, e.g. for the N5X the single model for a prediction time of 0ms average prediction errors were $ 17.01mm $ while for 66ms the errors increased to $ 19.96mm $.
This trend can be observed for all telephones, whether single or general model.
Same holds for the elliptic envelope areas where the enclosed area increases for increasing prediction times.
This suggests that the data we collected during our study is either not feasible for predicting touches into the future or that the amount of sensor movement patterns did not differ too much, meaning looking at the times to complete the touches (see \cref{sec:prepro}) participants tried to perform the touches as fast as possible to end the study leading to data consisting of a fairly high amount of fast touches and to an occlusion of slow touches. 
The fact that the general model performed better than the single models might be due to a better generalization of the data, leading to overall lower errors.
To return to the use cases from \cref{sec:intro}, our average errors (single models: 16.67mm$_{0ms}$ (SD = 2.47mm) to 18.82mm$_{66ms}$ (SD = 2.67mm) ; general model: 16.17mm$_{0ms}$ (SD = 1.88mm) to 18.05mm$_{66ms}$ (SD = 2.07mm)) will not allow content to be preloaded which requires clicking small icons or weblinks. 
However, our results could be applied to highlighting larger elements or preloading content from larger elements.

\subsection*{Limitations}
In this paper, we focused on one specific use task where users touched randomized targets on smartphones while sitting on a chair without armrests. 
The sensor samples generated during our study are specific for this use case and thereby do not cover ordinary smartphone usage and implied phone movement when e.g. walking, lying in bed, or operating the phone in a train.
An In-The-Wild study is open for future work, where participants are handed out phones which relentlessly record touches and sample the IMU.
The variety of touches during different activities would lead to a more generalizable data set for touch prediction.
The participants tried to complete the tasks as fast as possible because the study was exhausting for both the hand and the eyes.
This led to a high amount of sensor data from fast touches and a low percentage of slow touches.
Future work could explore the variety of different touches ranging from slow to fast touches  while sampling the IMU.
\begin{margintable}
	\vspace{-7cm}
	\centering
	\begin{tabularx}{1\marginparwidth}{XXd{4.2}d{4.2}d{4.2}}
		
		\toprule
		
		&\textbf{Model}&\multicolumn{1}{c}{\textbf{0ms}}&\multicolumn{1}{c}{\textbf{33ms}}&\multicolumn{1}{c}{\textbf{66ms}} \\
		\midrule
		
		\multirow{2}{*}{\textsc{S3}}   &\multirow{1}{*}{\textbf{S}}  &\cellcolor{green!25}766.92 & \cellcolor{green!25}864.35&1049.42       \\
		\cmidrule{2-5}
		&\multirow{1}{*}{\textbf{G}} &778.93&869.17&\cellcolor{green!25}1002.76     \\
		\midrule		
		\multirow{2}{*}{\textsc{S4}}   &\multirow{1}{*}{\textbf{S}}  &\cellcolor{green!25}1354.96&1588.03&1733.1       \\
		\cmidrule{2-5}
		&\multirow{1}{*}{\textbf{G}} &1376.14&\cellcolor{green!25}1482.92&\cellcolor{green!25}1656.89     \\
		\midrule
		\multirow{2}{*}{\textsc{N5X}}  &\multirow{1}{*}{\textbf{S}}  &\cellcolor{green!25}1413.76&1612.83&1972.12       \\
		\cmidrule{2-5}
		&\multirow{1}{*}{\textbf{G}} &1477.92&\cellcolor{green!25}1573.48&\cellcolor{green!25}1871.57     \\
		\midrule
		\multirow{2}{*}{\textsc{N6}}   &\multirow{1}{*}{\textbf{S}}  &1661.86&1814.29&2176.2       \\
		\cmidrule{2-5}
		&\multirow{1}{*}{\textbf{G}}&\cellcolor{green!25}1462.5&\cellcolor{green!25}1479.93&\cellcolor{green!25}1826.55\\							         		
		\bottomrule    
	\end{tabularx}%
	\caption[Ellipse areas]{\small Ellipse areas ($ mm^{2} $) from single models seen in \cref{fig:ell_single} and from general models seen in \cref{fig:ell_general}.}
	\label{tab:areas}
\end{margintable}

\section{Conclusion}
Two sentences wrap up what you have done. Than report what you achieved. 


% \begin{sidebar}
%  \textbf{Good Utilization of the Side Bar}
%
%  \textbf{Preparation:} Do not change the margin
%  dimensions and do not flow the margin text to the
%  next page.
%
%  \textbf{Materials:} The margin box must not intrude
%  or overflow into the header or the footer, or the gutter space
%  between the margin paragraph and the main left column.
%
%  \textbf{Images \& Figures:} Practically anything
%  can be put in the margin if it fits. Use the
%  \texttt{{\textbackslash}marginparwidth} constant to set the
%  width of the figure, table, minipage, or whatever you are trying
%  to fit in this skinny space.
%
%  \caption{This is the optional caption}
%  \label{bar:sidebar}
%\end{sidebar}
%
%
%
%\begin{marginfigure}
%    %\includegraphics[width=\marginparwidth]{cats}
%    \caption{In this image, the cats are tessellated within a square
%      frame. Images should also have captions and be within the
%      boundaries of the sidebar on page~\pageref{bar:sidebar}. Photo:
%      \cczero~jofish on Flickr.}
%    \label{fig:marginfig}
%\end{marginfigure}
%
%\begin{margintable}
%    \caption{A simple narrow table in the left margin
%      space.}
%    \label{tab:table2}
%    \begin{tabular}{r r l}
%      & {\small \textbf{First}}
%      & {\small \textbf{Location}} \\
%      \toprule
%      Child & 22.5 & Melbourne \\
%      Adult & 22.0 & Bogot\'a \\
%      \midrule
%      Gene & 22.0 & Palo Alto \\
%      John & 34.5 & Minneapolis \\
%      \bottomrule
%    \end{tabular}
%\end{margintable}
\newpage
\bibliography{bibliography}
\bibliographystyle{ACM-Reference-Format}

\end{document}
