\section{Related Work}
\label{sec:relw}
\todo{Story finden und dann umstellen/umschreiben}
Motivated by overcoming the lack of input modalities, related word presented touch prediction as a new interaction possibility on smartphones.
\citeauthor{MohdNoor2016} \cite{MohdNoor2016} presented \textit{28 Frames Later}, a system that predicts future touch positions on smartphones. 
Based on grip data they gathered from a total of 24 capacitive sensors built inside the BoD and on the laterals while performing touches on the touchscreen they built a machine learning model that was able to predict touch positions $ 200ms $ before the actual touch with an offset of $ 18mm $ to the actual touch position.
However their system required the 24 built on sensors, worked only on user-specific grip models, and their model could only make predictions in a maximum 3 $\times$ 3 grid.
\textit{BackXPress} by \citeauthor{Corsten2017} \cite{Corsten2017} is a novel interaction method that enhances touchscreen input by users applying pressure to the BoD with certain fingers to trigger specific actions on the screen.
Their built prototype did not confirm with the weight and form factor of ordinary phones and the system is only meant to be used in landscape orientation.
\citeauthor{Lochtefeld2015} \cite{Lochtefeld2015} used 

%We therefore structured our related work in the following parts: (1) existing solutions and research about touch prediction, and (2) \textbf{who the hell knows}.

%Motivate your project by reporting about related work and common goals.